\documentclass[12pt,aspectratio=1610]{beamer}

%\usefonttheme{serif}
\usepackage{helvet}
%\usetheme{Darmstadt}
%\usetheme{CambridgeUS}
%\usetheme{Frankfurt}
\usecolortheme{rose}
%\usetheme{Dresden}
\makeatother
\setbeamertemplate{footline}
{
  \leavevmode%
  \hbox{%
  \begin{beamercolorbox}[wd=.3\paperwidth,ht=2.25ex,dp=1ex,center]{author in head/foot}%
    \usebeamerfont{author in head/foot}\insertshortauthor
  \end{beamercolorbox}%
  \begin{beamercolorbox}[wd=.6\paperwidth,ht=2.25ex,dp=1ex,center]{title in head/foot}%
    \usebeamerfont{title in head/foot}\insertshorttitle
  \end{beamercolorbox}%
  \begin{beamercolorbox}[wd=.1\paperwidth,ht=2.25ex,dp=1ex,center]{date in head/foot}%
    \insertframenumber{} / \inserttotalframenumber\hspace*{1ex}
  \end{beamercolorbox}}%
  \vskip0pt%
}

\setbeamertemplate{itemize items}[triangle]
\setbeamertemplate{caption}[numbered]
\setbeamertemplate{theorems}[numbered]
\setbeamersize{text margin left=4.2mm,text margin right=7mm}

\usepackage[backend = biber, style = authoryear]{biblatex}
\addbibresource{bb.bib}

\title [Small title]{Event Title}
\subtitle{\textit{Research Title}} 
\author{Author}
\institute{Department}

\titlegraphic{
\vspace{0cm}\includegraphics[width=2.3cm, height = 2.3cm]{logo1.png}}


\date{}

\usepackage{amssymb}
\usepackage{graphicx}
\usepackage{amsmath}
\usepackage{calc}
\usepackage{hyperref}
\usepackage{grffile}
\usepackage{booktabs}
\usepackage{adjustbox}
\usepackage{threeparttable}
\newcommand{\explain}[2]{\underbrace{#1}_{\parbox{\widthof{\ensuremath{#1}}}{\footnotesize\raggedright #2}}}


\begin{document}


\theoremstyle{definition}
\newtheorem{proposition}{Proposition}

\setbeamertemplate{theorem begin}
{%
  \par\vskip\medskipamount%
  \begin{beamercolorbox}[colsep*=.75ex]{block title}
    \usebeamerfont*{block title}%
      \inserttheoremname
      \ifx\inserttheoremaddition\empty\else\ (\inserttheoremaddition)\fi%
  \end{beamercolorbox}%
  {\parskip0pt\par}%
  \ifbeamercolorempty[bg]{block title}
  {}
  {\ifbeamercolorempty[bg]{block body}{}{\nointerlineskip\vskip-0.5pt}}%
  \usebeamerfont{block body}%
  \vskip-.25ex\vbox{}%
}
\setbeamertemplate{theorem end}{}

\frame{
\titlepage }

\frame{
\begin{center}
\color[rgb]{0.2,0.2,0.698}{\LARGE{RELEVANCE}}\\
\end{center}
}

%%%%%%%%%%%%%%%%%%%%%%%%%%%%%%%%%%%%%%%%%%%%%%%%%%%%%%%%%%%%
%%%%%%%%%%%%%%%%% COLOR RED - 0.5686,0.0,0.1137 %%%%%%%%%%%%
%%%%%%%%%%%%%%%%% COLOR BLUE - 0.2,0.2,0.698 %%%%%%%%%%%%%%%
%%%%%%%%%%%%%%%%%%%%%%%%%%%%%%%%%%%%%%%%%%%%%%%%%%%%%%%%%%%%



%%%%%%%%%%%%%%%%%%% SINGLE ARGUMENT SLIDE %%%%%%%%%%%%%%%%%%%%%
\frame{
  \color[rgb]{0.2,0.2,0.698}{When is the last time you have opened you phone to Google something?}
    \begin{figure}
                    \centering
                    %\framebox{\includegraphics[width=0.60\linewidth, height=0.40\linewidth]{google.jpg}}
                    %\caption{\footnotesize{Marginal Effect of Interaction between Monthly Consumption Expenditure and Gender Group}}
    \end{figure}

}


\frame{
\frametitle{Introduction}
\small{\begin{itemize}
  \item Facebook - 3.8 billion users worldwide, 2.7 billion monthly active users, 1.8 billion daily active users - 3 hours per day .
\end{itemize}}
}

\frame{
\frametitle{Literature}
\small{\begin{enumerate}
  \item The existing literature seeks to examine the effect of social media and mobile phones on time spent on different activities and welfare effects .

\end{enumerate}}
}

\frame{
\begin{center}
\color[rgb]{0.5686,0.0,0.1137}{\large{Does higher mobile phone usage is linked to higher time spent on learning?}}\\
\end{center}
}


\frame{
\frametitle{Data and Measures}
\begin{enumerate}
\item Consumer Pyramid Survey (CPS) (Year 2019, Wave: September - December)
\item (X) \textbf{Mobile Phone Usage} - Monthly expenditure on mobile phones (recurring).
\item (Y) \textbf{Time spent on learning} - Average time spent on learning by members of the household.
\end{enumerate}
}


\frame{
\begin{center}
\color[rgb]{0.2,0.2,0.698}{\LARGE{EMPIRICAL STRATEGY}}\\
\end{center}
}

\frame[label = spec]{
\frametitle{Specification}
\small{\begin{itemize}
    \item The equation used to measure the effect of monthly expenditure on time spent on learning is as follows: \begin{equation*}
        Log(TL)_{ijt} = \beta_{0} + \beta_{1}Log(EXPM_{i}) + \gamma X_{i} + \mu_{j} + \nu_{t} + \varepsilon_{ijt}
    \end{equation*}
    where $Log(TL)_{ijt}$ is log of average time spent on learning by members of $i$ household, $j$ state $t$ in month $t$
    \item $Log(EXPM_{i})$ is the log monthly expenditure on mobile phones. $\beta_{1}$ is the primary effect of interest.
    \item $\mu_j$ and $\nu_j$ are state and time fixed effects.
    \item $\text{X}_{i}$ is the control variables. 
    \item $\varepsilon_{ijt}$ is the idiosyncratic error term that follows a $N(0,\sigma_{2})$.   
    \end{itemize}
\hyperlink{Appendix}{\beamerbutton{Appendix}}
}}


\frame{
\frametitle{Regression Results}
       %\centering \scriptsize{\input{table_REG}   
      }


%%%%%%%%%%%%%%%%%%%%%%%%%%%%%%%% FIGURE SLIDE %%%%%%%%%%%%%%%%%%%%%%%%%%%%%%%%

\frame{
\frametitle{Heterogenous Treatment Effects}
     
\begin{figure}
    \centering
    %\framebox{\includegraphics[width=0.60\linewidth, height=0.40\linewidth]{Heterogenous_treatment_effects by gender.jpg}}
    \caption{\footnotesize{Marginal Effect of Interaction between Monthly Consumption Expenditure and Gender Group}}
\end{figure}
        
}


\frame[label = post]{
\frametitle{Post Analysis Discussion}
\begin{itemize}
    \item Autocorrelation
    \item Heteroskedasticity
    \item Outliers
\end{itemize}
\hyperlink{residuals}{\beamerbutton{Appendix}}
}


\frame{
\frametitle{Possible Sources of Endogeneity}
\begin{itemize}
        \item \textit{Measurement Error and Reporting Bias}: Expenditure and Quantifying time spent on learning.
        \item \textit{Omitted Variable Bias}: Baseline values of Y (Habit forming behaviour), Unmeasurables - curiousity to learn, district level infrastructure, etc.
        \item \textit{Reverse Causality}: What if more time spent on learning leads to more expenditure on mobile phones? (Rational decision - ROI)
\end{itemize}  
}

\frame{
\frametitle{Future Directions}
\begin{itemize}
        \item More village level controls variables: Other data sources.
        \item Use of instruments variables to address endogeneity (two i have tested).
\end{itemize}  
}



\frame{
\printbibliography
}



\frame{
        \begin{center}
            \color[rgb]{0.2,0.2,0.698}{\fontsize{40}{40}\selectfont Thank You}\\
        \end{center}
}

%%%%%%%%%%%%%%%%%%%%%%%%%%%%%%%%%%%%%%%%%%%%%%%%%%%%%%%%%%%%%%%%%%%%%%%%%%%%%%%%%%%%%
%%%%%%%%%%%%%%%%%%%%%%%%%%%%%%%%     APPENDIX    %%%%%%%%%%%%%%%%%%%%%%%%%%%%%%%%%%%%
%%%%%%%%%%%%%%%%%%%%%%%%%%%%%%%%%%%%%%%%%%%%%%%%%%%%%%%%%%%%%%%%%%%%%%%%%%%%%%%%%%%%%

\frame[label=Appendix]{
        \begin{center}
            \color[rgb]{0.2,0.2,0.698}{\LARGE{Appendix}}\\
        \end{center}
}

\frame{
        \frametitle{Scatter Plot of Y and X}
     
                \begin{figure}
                    \centering
                    %\framebox{\includegraphics[width=0.60\linewidth, height=0.40\linewidth]{xyscatter.jpg}}
                    %\caption{\footnotesize{Marginal Effect of Interaction between Monthly Consumption Expenditure and Gender Group}}
                \end{figure}
                \hyperlink{spec}{\beamerbutton{Back}}
        
}


\frame[label = residuals]{
        \frametitle{Normality of Residuals}
     
                \begin{figure}
                    \centering
                    %\framebox{\includegraphics[width=0.60\linewidth, height=0.40\linewidth]{qnorm_.jpg}}
                    %\caption{\footnotesize{Marginal Effect of Interaction between Monthly Consumption Expenditure and Gender Group}}
                \end{figure}
                \hyperlink{post}{\beamerbutton{Back}}
}

\frame{
        \frametitle{Residual Plot}
     
                \begin{figure}
                    \centering
                    %\framebox{\includegraphics[width=0.60\linewidth, height=0.40\linewidth]{residual_plot.jpg}}
                    %\caption{\footnotesize{Marginal Effect of Interaction between Monthly Consumption Expenditure and Gender Group}}
                \end{figure}
                \hyperlink{post}{\beamerbutton{Back}}
}






\end{document}
